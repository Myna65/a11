Synthesis of~\cite{highestRankOfAn},~\cite{A12PolytopesRank},~\cite{highRankAlternating},~\cite{bookIncidenceGeometry} and~\cite{cprGraph}.

\section{Abstract Polytopes}

\begin{definition}[Partially ordered set]
  A \textit{partially ordered set} is a pair $(R,\le)$ with $R$ a set and $\le$ a relation between $R$ and $R$ such that
  \begin{enumerate}
    \item $F \le F$ ($F \in R$)
    \item if $F \le G$ and $G \le F$, we have $F=G$ ($F,G \in R$)
    \item if $F \le G$ and $G \le H$, we have $F \le H$ ($F,G,H \in R$)
  \end{enumerate}
\end{definition}

\begin{definition}[Face of a partially ordered set]
  In the context of abstract polytopes, we denote elements of the poset, \textit{faces}.
\end{definition}

\begin{definition}[Ranked partially ordered set]
  A poset is ranked if there exists a function $\rank : R \to \mathbb Z$ such that
  \begin{enumerate}
    \item if $F < G$, then $\rank(F) < \rank(G)$ $(F, G \in R)$
    \item if $F < G$ and there exists no $H \in R$ such that $F < H < G$, then $\rank(G) = \rank(F) + 1$ $(F, G \in R)$.
  \end{enumerate}
\end{definition}

\begin{definition}[$i$-face]
  If a face have a rank equals to $i$, we will call this face a $i$-face.
\end{definition}

\begin{definition}[Totally ordered set]
  A \textit{totally ordered set} is a poset such that $\forall x,y \in R$, we have $x \le y$ or $y \le x$.
\end{definition}

\begin{definition}[Flag of a partially ordered set]
  A \textit{flag} of a partially ordered set is a maximal totally ordered which is a subset of the poset.
\end{definition}

\begin{definition}[Abstract $d$-polytope]
  An \textit{abstract $d$-polytope} $\mathcal P$ is a ranked partially ordered set of \textit{faces}. Such that
  \begin{enumerate}
    \item $\mathcal P$ contains two improper faces, a least face $F_{-1}$ and a greatest face $F_d$.
    \item The flags of $\mathcal P$ contains $d + 2$ faces (includind the 2 improper faces)
    \item $\mathcal P$ is strongly connected (see below)
    \item Let $F, G$ be two faces of $\mathcal P$ such that $\rank(F) = i - 1$ and $\rank(G) = i + 1$. Then there exists exactly 2 $i$-faces $H$ such that $F < H < G$. This is called the diamond property.
  \end{enumerate}
\end{definition}

\begin{definition}[Section of a poset]
  For any two faces $F, G$ with $F \le G$, we define the \textit{section} \[
    F/G = \{H | H \in R, F \le H \le G\}
  \]
\end{definition}

\begin{property}
  Every section of a polytope is a polytope
\end{property}

\begin{definition}[Connected poset]
  A poset which satisfy the two first properties of a polytope is said \textit{connected} if $d \le 1$ or if, for any proper faces $F, G$, there exists a sequence of proper faces $F = H_0, H_1, \dots, H_{k-1}, H_k = G$. such that $H_i$ and $H_{i+1}$ are compararable for every $i$.
\end{definition}

\begin{definition}[Strictly connected poset]
  A poset is \textit{strictly connected} if every section (including the whole poset) is connected.
\end{definition}

\begin{definition}[Adjacent flags]
  Two flags of a $d$-polytope are said adjacent if they differ by exactly one face.
\end{definition}

\begin{definition}[$i$-adjactent flag]
  Let $\Phi$ be a flag, we know by the diamond property that there exists exactly a flag that share all faces execpt the face of rank $i$ with $\Phi$. We call this face the $i$-adjacent flag of $\Phi$ and we denote it $\Phi^i$.
\end{definition}

\begin{property}
  \[
    \left(\Phi^i\right)^i = \Phi
  \]
\end{property}

\begin{property}
  \[
    \left(\Phi^i\right)^j = \left(\Phi^j\right)^i \quad \text{if} \ |i-j| > 1.
  \]
\end{property}

\begin{definition}[Equivar polytope]
  A $d$-polytope is said \textit{equivar} if for all $i = 1, 2, \dots, d-1$ there is an integer $p_i$ such that any section $G/F$ where $F$ is $(i-2)$-face and $G$ is a $(i+1)$-face.
\end{definition}

\begin{definition}[Schläfli type]
  If $\mathcal P$ is equivelar, we say that ha have a \textit{Schläfli type} of $\{p_1, \dots, p_{d-1}\}$.
\end{definition}

\begin{definition}[Regular polytope]
  A polytope is regular if it's automoprhism group have exactly one orbit over the flags.
\end{definition}

\begin{property}
  Equivalently, a polytope is regular if for some flag $\Phi$, and each $i$, there exists a unique involutory automorphism $\rho_i$ such that $\Phi_i = \Phi \rho_i$. In fact this property holds for every flag.
\end{property}

\begin{definition}[Base flag]
  We choose a fixed flag of a polytope and we call it the \textit{base flag}.
\end{definition}

\begin{property}
  For a regular $d$-polytope with base flag $\Phi$, it's group $\Gamma(\Phi)$ is generated by the involutions $\rho_i$ defined below.
\end{property}

\begin{definition}[Distinguised generator]
  The $\rho_i$ are called the \textit{distinguished generators}.
\end{definition}

\section{Groups}

\begin{definition}[String group generated by involutions]
  Let $G$ be a group and $S = \{\rho_0, \dots, \rho_n\}$ be a set of involutions which generate $G$ such that
  \begin{center}
    $\forall i,j$ with $| i - j| > 1$ $\rho_i$ and $\rho_j$ commute.
  \end{center}
  We call the pair $(G,S)$ a \textit{string group generated by involutions}
\end{definition}

\begin{definition}[string C-group]
  Denote $\Gamma_I$ the group generated by $\{\rho_i : i \in I\}$ where $I \subseteq \{0\dots r-1\}$. A sggi $(G,S)$ is a \textit{string C-group} if it satisfies the intersection property:
  \begin{center}
    $\Gamma_I \cap \Gamma_J = \Gamma_{I \cap J}$ where $I, J \subseteq \{0\dots r-1\}$. We denote $\Gamma = (G,S)$ for string C-groups.
  \end{center}
\end{definition}

\begin{definition}[Rank of a string C-group]
  The \textit{rank of a string C-group} $\Gamma = (G,S)$ is $|S|$.
\end{definition}

\begin{remark}[Group $\Gamma$]
  We will talk about the group $\Gamma$ where $\Gamma = (G,S)$ to talk about the group $G$.
\end{remark}

\begin{remark}[Rank of $G$]
  We will talk about the rank of the group $G$ as the maximum size of a set $S$ of involutions such that $\Gamma = (G,S)$.
\end{remark}

\begin{property}
  The group of automorphisme of a regular polytope and the distinguished generators forms a string C-group and the Schälfli symbol of this polytope is $\{p_1, \dots, p_d\}$
\end{property}

\begin{property}
  A $d$-polytope can be uniquely constructed from a string $C$-group.
\end{property}

\section{Properties of groups}

\begin{property}
  Let $\Gamma = \langle \rho_0, \dots, \rho_{r-1} \rangle$ be a sggi such that $\Gamma_0, \Gamma_{r-1}$ are string C-groups and $\Gamma_0 \cap \Gamma_{r-1} = \Gamma_{0,r-1}$ then $\Gamma$ is a string C-group.
\end{property}

\begin{property}
  Let $\Gamma$ be a sggi, if $\Gamma_0$ and $\Gamma_{r-1}$ are string C-group, $\rho_{r-1} \notin \Gamma_{r-1}$ and $\Gamma_0$ is maximal
\end{property}

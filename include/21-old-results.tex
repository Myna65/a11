\section{Basic results v2}

\begin{proposition}
  If a vertex is linked with three edges that does not share the other end, then two of those three edges are part of an alternating square.
\end{proposition}

\begin{proof}
  Among the three edges, at least two are not adjacent and thus, they need to be part of an alternating square.
\end{proof}

\begin{corollary}
  If an vertex is not part of any alternating square, then it it connected to at most two other vertices.
\end{corollary}

\begin{proposition}
  In a connected CPR graph over more than two vertices, there no triple edge outside of an alternating square and there is no double edge such that the difference between indices is not 2.
\end{proposition}

\begin{proof}
  Each end of the triple edge is not part of an alternating square and so must not be connected to more than 2 vertices. But the graph is connected, so at least one end of the graph must be connected to more than one vertex. So one end must be connected to two vertices but then the edge that connect this vertex to the new one must be adjacent to all three edges. But that is impossible.
\end{proof}

\begin{corollary}
  \label{sequence-connection}
  If a component of the permutation representation graph is composed with alternating squares. If this sequence cannot be extended it must be connected to other component of the graph by single edge.
\end{corollary}

\begin{proposition}
  \label{square-connection}
  An alternating square can be connected to a simple edge (no part of an alternating square) only if it have two adjacent simple edges with a difference in indices of exactly 2. The index of the edge used to connect the square must be between the indices of the edges of the square.
\end{proposition}

\begin{lemma}
  \label{chain-consecutive}
  In a chain, the indices must be consecutive.
\end{lemma}



\section{Basic results}

\paragraph{}
Dans la suite, nous noterons les carrés alternés grâce au deux involutions qui les composent. Un carré alterné avec des arêtes $\rho_0$ et $\rho_2$ sera noté $[\rho_0, \rho_2]$.

\begin{definition}
  Une suite de carrés alternés adjacents est un suite finie dans laquelle chaque carré alterné est adjacent avec son précédesseur et son successeur s'ils existent.
\end{definition}


\begin{lemma}
  La parité de la taille d'une suite de carrés alternés est toujours la même que la partité d'une suite monotone qui admet les deux même extrémités.
\end{lemma}

\begin{lemma}
  \label{lemma-continue-alternating-square}
  Si on a un carré alterné dont les indices différent de plus 2 alors le seul moyen de l'étendre est d'utiliser un autre carré alterné.
\end{lemma}

\begin{corollary}
  Si nous travaillons sur un nombre impair de points et si nous avons un carré alterné dont la différence des indices est de plus de 2, alors toute extension de ce carré alterné à tous les points contient une suite de carrés alternés qui comprend un carré dont la différence des indices est exactement 2.
\end{corollary}

\begin{proof}
  Chaque fois que nous étendons avec un carré alterné nous ajoutons deux points. Si nous avons un nombre impair de points alors nous ne pouvons pas utiliser que des carrés alternés pour les relier. Mais si nous avons un carré alterné dont la différence des indices est de plus de deux, nous devons forcément l'étendre à un autre carré alterné. Et ainsi de suite jusqu'à avoir tous les points ou à arriver à un carré alterné dont la différence des indice ne vaut plus que deux. Le premier cas est impossible donc c'est forcément le second qui arriver.
\end{proof}

\begin{lemma}
  La taille d'une suite monotone partant du carré alterné $[i, j]$ et arrivant à une carré donc la différence des indices est 2 est $|i - j| - 2$.
\end{lemma}

\begin{lemma}
  \label{parity-sequence-squares}
  La taille d'une suite allant d'un carré alterné dont la différence des indices est 2 vers un autre carré alterné qui satisfait le même propriété est impaire.
\end{lemma}

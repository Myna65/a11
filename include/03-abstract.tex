\cleardoublepage{}

\addcontentsline{toc}{chapter}{Abstract and keywords}
\thispagestyle{plain}

\hspace{3cm}

\begin{quote}

\vfill{}

\begin{otherlanguage}{french}

\begin{center}
  \textbf{Mots clés}
\end{center}

Polytopes abstraits réguliers, String C-groups, Groupes alterné, Permutation representation graphs.

\begin{center}
  \textbf{Résumé}
\end{center}


\paragraph{}
Dans~\cite{highestRankOfAn} les auteurs ont trouvé le rang maximal d'un polytope pour les groupes alternés. À l'aide de l'ordinateur, ils se sont rendus compte que l'ensemble des rangs admissibles pour des polytopes sur le groupe $A_{11}$ est $\{3,6\}$. Actuellement, c'est le seul groupe connu où l'ensemble des rangs admissibles n'est pas un intervalle. Nous essayons de comprendre pourquoi cet ensemble n'est pas un intervalle, en particulier nous prouvons pourquoi pourquoi il n'y a aucun polytope de rang 4 ou 5 dont le groupe d'automorphisme est $A_{11}$.

\end{otherlanguage}

\vfill

\begin{center}
  \textbf{Keywords}
\end{center}

Abstract regular polytopes, String C-groups, Alternate groups, Permutation representation graphs.

\begin{center}
  \textbf{Abstract}
\end{center}

\paragraph{}
In~\cite{highestRankOfAn} the authors found the maximal rank for a polytope on alternate groups. With the help of a computer, they realized that the set of admissible rank for polytopes on the group $A_{11}$ is $\{3,6\}$. Currently, this is the only known group such that this set is not an interval. We try to understand why this set is not an interval, in particular we proved why there are no polytopes of rank 4 or 5 with automorphism group $A_{11}$.

\vfill

\end{quote}

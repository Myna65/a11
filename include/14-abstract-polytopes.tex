\subsection{Abstract Polytopes}

\paragraph{}
This section uses~\cite{incidenceGeometry} and \cite{abstractRegularPolytopes}.

\paragraph{}
Given a face lattice it is not possible to not known whether there exists an euclidean polytope that admit this face lattice or not. Therefore it is not possible to modify or to generate face lattice. So we define the concept of \textit{abstract polytope}. An abstract polytope is a face lattice which satisfies the four major properties of the face lattice of polytopes but without asking them to have a representation in an euclidean space. With abstract polytopes, only the combinatorial structure is kept.

\begin{definition}[Abstract $d$-polytope]
  An \textit{abstract $d$-polytope} $\mathcal P$ is a ranked lattice. The elements of the lattice are called \textit{faces}. This lattice must be such that
  \begin{enumerate}
    \item $\mathcal P$ contains two improper faces, a least face $F_{-1}$ and a greatest face $F_d$
    \item The flags of $\mathcal P$ contains $d + 2$ faces (including the 2 improper faces)
    \item $\mathcal P$ is strongly connected
    \item $\mathcal P$ satisfies the diamond property.
  \end{enumerate}
\end{definition}

\paragraph{}
The main subject of this work is abstract polytope. Therefore the word "polytope" will be used to talk about abstract polytopes and not about euclidean polytopes.

\begin{definition}[Flags of polytope]
  The flags of a polytope are the maximal chains of the underlying poset. The set of all flags is denoted $\mathcal F(P)$.
\end{definition}

\subsection{Morphisms}

\begin{definition}[Homomorphism of polytopes]
  \index{homomorphism}
  Let $\mathcal P$ and $\mathcal Q$ be two abstract polytopes. A homomorphism $\varphi$ from $\mathcal P$ to $\mathcal Q$ is a map between the faces of $\mathcal P$ and $\mathcal Q$ such that, for $F, G \in P$ with $F \le_{\mathcal P} G$ then $F\varphi \le_{\mathcal Q} G\varphi$.
\end{definition}

\begin{definition}[Isomorphism of polytopes]
  \index{isomorphism}
  An isomorphism of polytopes is a bijection $\varphi$ such that $\varphi$ and $\varphi^{-1}$ are homomorphisms.
\end{definition}

\begin{definition}[Automorphism of a polytope]
  \index{automorphism}
  An automorphism of a polytope $\mathcal P$ is an isomorphism between $\mathcal P$ and itself.
\end{definition}

\paragraph{}
The set of all automorphisms of a polytope $\mathcal P$ forms a group called the automorphism group of the polytope. This group is denoted $\Gamma(\mathcal P)$.

\subsection{Regularity}

\begin{definition}
  An abstract regular polytope is a polytope such that $\Gamma(\mathcal P)$ is transitive on $\mathcal F(\mathcal P)$.
\end{definition}

\begin{theorem}
  The automorphism group of an abstract regular polytope is a string C-group.
\end{theorem}

\subsection{Generators}

\begin{proposition}
  \label{adjacent-must-not-commute}
  If two consecutive generators commute then the generated group cannot be almost simple.
\end{proposition}

\section{Abstract Polytopes}

\paragraph{}
This section uses~\cite{incidenceGeometry} and \cite{abstractRegularPolytopes}.

\paragraph{}
The problem with polytopes is that, as of today, given a face lattice we can not known whether it's a polytope or not. And so we cannot work on face lattice which in annoying. We want to work on this face lattice and so we define an \textit{abstract polytope} as a face lattice which satisfy all major properties of face lattice of polytope but without asking them to have a representation in an euclidean space. With abstract polytopes, we only keep the combinatorial structure.



\begin{definition}[Face of a partially ordered set]
  In the context of abstract polytopes, we denote elements of the poset, \textit{faces}.
\end{definition}

\begin{definition}[$i$-face]
  If a face have a rank equals to $i$, we will call this face a $i$-face.
\end{definition}

\begin{definition}[Flag of a partially ordered set]
  A \textit{flag} of a partially ordered set is a subset of the poset which is maximal totally.
\end{definition}

\begin{definition}[Abstract $d$-polytope]
  An \textit{abstract $d$-polytope} $\mathcal P$ is a ranked partially ordered set of \textit{faces}. Such that
  \begin{enumerate}
    \item $\mathcal P$ contains two improper faces, a least face $F_{-1}$ and a greatest face $F_d$.
    \item The flags of $\mathcal P$ contains $d + 2$ faces (including the 2 improper faces)
    \item $\mathcal P$ is strongly connected (see below)
    \item Let $F, G$ be two faces of $\mathcal P$ such that $\rank(F) = i - 1$ and $\rank(G) = i + 1$. Then there exists exactly 2 $i$-faces $H$ such that $F < H < G$. This is called the diamond property.
  \end{enumerate}
\end{definition}

\begin{definition}[Section of a poset]
  For any two faces $F, G$ with $F \le G$, we define the \textit{section} \[
    F/G = \{H | H \in R, F \le H \le G\}
  \]
\end{definition}

\begin{property}
  Every section of a polytope is a polytope
\end{property}

\begin{definition}[Connected poset]
  A poset which satisfy the two first properties of a polytope is said \textit{connected} if $d \le 1$ or if, for any proper faces $F, G$, there exists a sequence of proper faces $F = H_0, H_1, \dots, H_{k-1}, H_k = G$. such that $H_i$ and $H_{i+1}$ are compararable for every $i$.
\end{definition}

\begin{definition}[Strictly connected poset]
  A poset is \textit{strictly connected} if every section (including the whole poset) is connected.
\end{definition}

\begin{definition}[Adjacent flags]
  Two flags of a $d$-polytope are said adjacent if they differ by exactly one face.
\end{definition}

\begin{definition}[$i$-adjacent flag]
  Let $\Phi$ be a flag, we know by the diamond property that there exists exactly a flag that share all faces execpt the face of rank $i$ with $\Phi$. We call this face the $i$-adjacent flag of $\Phi$ and we denote it $\Phi^i$.
\end{definition}

\begin{property}
  \[
    \left(\Phi^i\right)^i = \Phi
  \]
\end{property}

\begin{property}
  \[
    \left(\Phi^i\right)^j = \left(\Phi^j\right)^i \quad \text{if} \ |i-j| > 1.
  \]
\end{property}

\begin{definition}[Equivar polytope]
  A $d$-polytope is said \textit{equivar} if for all $i = 1, 2, \dots, d-1$ there is an integer $p_i$ such that any section $G/F$ where $F$ is $(i-2)$-face and $G$ is a $(i+1)$-face.
\end{definition}

\begin{definition}[Schläfli type]
  If $\mathcal P$ is equivelar, we say that ha have a \textit{Schläfli type} of $\{p_1, \dots, p_{d-1}\}$.
\end{definition}

\begin{definition}[Regular polytope]
  A polytope is regular if it's automoprhism group have exactly one orbit over the flags.
\end{definition}

\begin{property}
  Equivalently, a polytope is regular if for some flag $\Phi$, and each $i$, there exists a unique involutory automorphism $\rho_i$ such that $\Phi_i = \Phi \rho_i$. In fact this property holds for every flag.
\end{property}

\begin{definition}[Base flag]
  We choose a fixed flag of a polytope and we call it the \textit{base flag}.
\end{definition}

\begin{property}
  For a regular $d$-polytope with base flag $\Phi$, it's group $\Gamma(\Phi)$ is generated by the involutions $\rho_i$ defined below.
\end{property}

\begin{definition}[Distinguised generator]
  The $\rho_i$ are called the \textit{distinguished generators}.
\end{definition}

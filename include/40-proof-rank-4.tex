\chapter{Proof for rank 4}

\paragraph{}
Nous allons maintenant étudier les différents cas en rang 4. Nous pouvons avoir deux, trois ou quatre 4-transpositions. Étudions ces cas successivement.

\begin{theorem}
  $\rho_1$ et donc $\rho_2$ ne peuvent être des 2-transposition.
\end{theorem}

\begin{proof}
  $\rho_2$ ou $\rho_3$ est une 4-transposition. Donc $\rho_0$ doit partager des sommets avec au moins une d'entre elle.

  \paragraph{}
  Si c'est $\rho_2$ qui est une 4-transposition, $\rho_0$ et $\rho_3$ doivent partager un sommet. Dès lors, elles doivent former soit un carré alterné $[\rho_0, \rho_3]$ soit une arête double $(\rho_0, \rho_3)$.

  \paragraph{}
  Dans le premier cas, ce carré alterné devra être adjacent à un autre carré alterné $[\rho_0, \rho_2]$, dans le second cas, on a le choix entre ce carré et un carré du type $[\rho_1, \rho_3]$. Dans les deux cas, nous avons utilisés toutes nos arêtes exédentaires, la suite devra donc être linéaire. En fonction des cas que nous avons faits, il y a trois possibiliés:

  \begin{itemize}
    \item 1 $\rho_0$, 2 $\rho_1$, 2 $\rho_2$, 0 $\rho_3$
    \item 2 $\rho_0$, 2 $\rho_1$, 2 $\rho_2$, 1 $\rho_3$
    \item 3 $\rho_0$, 0 $\rho_1$, 4 $\rho_2$, 0 $\rho_3$
  \end{itemize}
\end{proof}

\paragraph{}
Le troisième cas est clairement impossible car, en partant d'une arête $\rho_2$, nous ne pourons jamais atteindre $\rho_0$ de manière linéaire.

\paragraph{}
Dans les deux premiers cas, nous devons continuer avec des arêtes $\rho_1$ donc le trois

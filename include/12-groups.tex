\section{Groups}

\subsection{Reminders from groups theory}

\begin{definition}[Group]
  \index{group}
  A group $G$ is a pair $(E, \cdot)$ where $E$ is a set and $\cdot$ is an operator $\cdot: E \times E \to E$. A group must satisfy the following properties:
  \begin{itemize}
    \item \textbf{Associativity} $(a \cdot b) \cdot c = a \cdot (b \cdot c)\ \forall a,b,c \in G$.
    \item \textbf{Identity element} There exists an element of $G$ (called $1$) such that $a \cdot 1 = a = 1 \cdot a \ \forall a \in G$.
    \item \textbf{Inverse element} $\forall a \in G$ there exists $a^{-1} \in G$ such that $a \cdot a^{-1} = 1 = a^{-1} \cdot a$.
  \end{itemize}
\end{definition}

\begin{definition}[Generating set of a group]
  \index{generating set}
  \index{generator}
  It is said that a set generates a group if all elements of the group can be written as a product of elements of this set. The elements of this set are called \textit{generators}. Conversely, no element outside the group can be written as a product of the generators.
\end{definition}

\subsection{Permutation groups}

\paragraph{}
The followings statements about permutation groups come from~\cite{cameronPermutationGroups}.

\begin{definition}[Permutation]
  \index{permutation}
  A permutation of $S$ is a bijection between $S$ and itself. It can be seen as a reordering of the elements of $S$.
\end{definition}

\paragraph{}
Instead of the using the functional notation, the permutation will be written on the right on the composing read from the left to right. The application of the permutation $g$ to a point $\alpha$ is noted $\alpha g$ and not $g(\alpha)$. This make the notation lighter and the reading easier. So $\alpha gh$ will be written instead of $h(g(\alpha))$.

\paragraph{}
One common way to write permutation is the following notation.

\[
  g =
  \left(
    \begin{array}{ccccc}
      1 & 2 & 3 & 4 & 5\\
      2 & 5 & 3 & 4 & 1
    \end{array}
  \right)
\]

\paragraph{}
Here $1g = 2$ and $5g = 1$.

\begin{definition}[Fixed point]
  \index{fixed point (permutation)}
  A fixed point of a permutation $g$ is a point $\alpha$ such that $\alpha g = \alpha$.
\end{definition}

\begin{definition}[Transposition]
  \index{transposition}
  A transposition is a permutation that swap two points and let all others points fixed.
\end{definition}

\paragraph{}
A transposition of the point $\alpha$ and $\beta$ is written $(\alpha\ \beta)$.

\begin{definition}[Cycle]
  \index{cycle (permutation)}
  \index{cyclic permutation}
  A \textit{cycle} or \textit{cyclic permutation} is a permutation which maps elements in a cyclic fashion. All other points are fixed.
\end{definition}

\paragraph{}
We note a cycle $(c_1\ c_2\ \dots\ c_n)$ where the $c_i$ are the elements moved by the cyclic permutation.

\paragraph{}
The term \textit{cyclic permutation} is used to describe a full permutation and the term \textit{cycle} is used to describe a part of a permutation. But the meaning remains the same.

\begin{proposition}
  Every cycle can be written as a product of transposition.
\end{proposition}

\begin{proof}
  \[
    (c_1\ c_2\ c_3\ \dots\ c_n) = (c_1\ c_2)(c_1\ c_3)\dots(c_1\ c_n)
  \]
\end{proof}

\begin{proposition}
  Every permutation can be written as a product of a cycles.
\end{proposition}

\begin{proof}
  For every non fixed-point $\alpha_0$, we can find a sequence defined by $\alpha_{i+1} = \alpha_i g$. There must be a $j \neq 0$ such that $\alpha_j = \alpha_0$. $g$ is a bijection between the same set. And so it's a bijection between two finite sets of the same size. But if the sequence does not loop back to its first element, it can't loop at all because it's a bijection. So it must be infinite and it's a contradiction.
  % TOO COMPLICATED
\end{proof}

\begin{corollary}
  Every permutation can be written as a product of transposition.
\end{corollary}

\paragraph{}
The new notation for permutation can now be introduced. Permutation will now be written as a product of cycle. This new notation does not include the total number of points. This must be clear in the context.

\begin{definition}[Symmetric group]
  \index{symmetric group}
  Let $\Omega$ be a set, the group of all permutations of $\Omega$ is a called the symmetric group of $\Omega$ and is denoted $\Sym(\Omega)$. Often we do not care about the elements of $\Omega$, the only important thing about $\Omega$ is its size because the structure of $\Sym(\Omega)$ is only determined by this parameter. Therefore, we will mostly denote $S_n$ the symmetric group over an set of $n$ elements.
\end{definition}

\begin{definition}[Permutation group]
  \index{permutation}
  A permutation group on a set $\Omega$ (resp. a set of $n$ points) is a subgroup of $\Sym(\Omega)$ (resp. $S_n$).
\end{definition}

\subsection{Transitivity}

\begin{definition}[$k$-transitive group]\index{k-transitive group}
  A permutation group $G$ on $\Omega$ is called \textit{$k$-transitive} if for every pair of $k$-uple $((\alpha_1, \dots, \alpha_k), (\beta_1, \dots, \beta_k)$ with $\alpha_i, \beta_i \in \Omega$ there exists at least one $g \in G$ such that $\alpha_i g = \beta_i \  \forall i$.
\end{definition}

\begin{definition}[Sharp $k$-transitive group]\index{sharp k-transitive group}
  A permutation group $G$ on $\Omega$ is called \textit{sharp $k$-transitive} if for every pair of $k$-uple $((\alpha_1, \dots, \alpha_n), (\beta_1, \dots, \beta_n)$ with $\alpha_i, \beta_i \in \Omega$ there exists exactly one $g \in G$ such that $\alpha_i g = \beta_i \  \forall i$.
\end{definition}

\subsection{Primitivity}

\begin{definition}[Congruence]
  A congruence $\sim$ is an equality relation on $\Omega$ such that $\alpha \sim \beta \Leftrightarrow g\alpha \sim g\beta \ \forall g \in G$.
\end{definition}

There are two trivial congruences on all permutation groups:
\begin{itemize}
  \item Equality
  \item Universal relation i.e. $\alpha \sim \beta \ \forall \alpha, \beta \in \Omega$.
\end{itemize}

\begin{definition}[Primitive group]
  A group is called \textit{primitive} if it does not have any non-trivial congruence.
\end{definition}

\begin{property}
  Every 2-transitive permutation group is primitive
\end{property}

\begin{proof}
  TO DO?
\end{proof}

\subsection{String groups generated by involutions}

\begin{definition}[Involution]
  An element $g$ of a group is called an \textit{involution} if $g^2 = \id$.
\end{definition}

\begin{definition}[Indexed generated group]
  Let $G$ be a group and let $S = \{\rho_0, \dots, \rho_{r-1}\}$ be an indexed set of generators.
  We call the pair $(G,S)$ an \textit{indexed generated group}. It will be denoted $\Gamma$.
\end{definition}

\begin{definition}[Rank of an indexed generated group]
  The rank of an indexed generated group is the number of generators, i.e. the size of $S$.
\end{definition}

\begin{definition}[Group generated by involutions]
  Let $\Gamma$ be an indexed generated group. If all the generators of this group are involutions, then this group is called a \textit{group generated by involutions}.
\end{definition}

\begin{definition}[String property]
  Let $\Gamma$ be an indexed generated group of rank $r$. If for all $0 \le i < r-1$, $\rho_i$ and $\rho_{i+1}$ commutes ($\rho_i \rho_{i+1} = \rho_{i+1}\rho_i$), then this indexed generated group has the \textit{string property}.
\end{definition}

\begin{definition}[String group generated by involution]
  If a group generated by involutions has the string property, we call it a \textit{string group generated by involution} or \textit{sggi}.
\end{definition}

\subsection{String C-groups}

\paragraph{}
Let $\Gamma = (G,S)$ be an indexed generated group. We denote $\Gamma_I$ with $I \subseteq S$ the permutation group by $I$.

\begin{definition}[Intersection property]
  Let $\Gamma = (G,S)$ be a indexed generated group. If for all $I, J \subseteq S$, we have that $\Gamma_I \cap \Gamma_J = \Gamma_{I \cap J}$, we say that $\Gamma$ satisfies the intersection property.
\end{definition}

\begin{definition}[String C-group]
  If a sggi satisfies the intersection property, we will say that is a $\textit{string C-group}$.
\end{definition}

\chapter*{Introduction}
\addcontentsline{toc}{chapter}{Introduction}

\paragraph{}
The beginning of the study of regular polyhedra goes back to the Greeks. In particular, Plato is credited with the classification of regular polyhedra of dimension three, hence their name: Platonic solids.

\paragraph{}
For two millennia, little progress has been made with the notable exception of the classification of star polyhedra of dimension three by Kepler at the beginning of the 17th century. It was not until the 19th century that scientists were interseted in polyhedra of dimension larger than three.

\paragraph{}
Many results have from that moment been found. These results were valid for polytopes of any dimension. However, these results also apply to a larger set of objects that are not polytopes. These object have been named \textit{abstrat polytopes}.

\paragraph{}
The study of these abstract polytopes quickly focused on familles of polytopes with certains properties of symmetry: regular abstrat polytopes as well as chiral abstract polytopes. The high degree of symmetry makes these abstrat polytopes interesting because their automorphism group is large. These automorphism group have thus been studied by means of groups of permutations.

\paragraph{}
The question ended up being reversed and it became more intersting to find all the polytopes admitting a given automorphism group. The research quickly focused on almost simple groups.

\paragraph{}
In the second half of the 20th century, a correspondance was established between regular abstrat polytopes and what we called string C-group representations. This simplified the study of abstract polytopes. Indeed these string C-group representations are representations of the automorphism group of an abstract regular polytopes that further completly define this abstract polytope. This allowed the use of group theory results in the study of abstrat polytopes.

\paragraph{}
An atlas of all abstract polytopes admitting small almost simple groups is available in~\cite{atlasPolytopes}.

\begin{otherlanguage}{french}

\paragraph{}
Concernant les grands groupes, la plupart des recherches se sont concentrées sur $S_n$ et $A_n$. Pour $S_n$, il a été prouvé dans~\cite{highRankSym} qu'il existe exactement un polytope de rang $n-1$ qui admet $S_n$ comme groupe d'automorphismes pour $n \ge 5$. De même il a été prouvé qu'il existe exactement un polytope abstrait de rang $n-2$ qui admet $S_n$ comme groupe de symétrie pour $n \ge 7$. Plus récemment, il a été prouvé dans~\cite{leemansTransactions} qu'il existe exactement 7 polytopes abstraits de rang $n-3$ qui ont comme groupe d'automorphismes $S_n$ pour $n \ge 9$. Il est actuellement conjecturé que cette propriété est vraie pour 9 polytopes de rang $n-3$ pour $n \ge 11$. C'est une question actuellement ouverte.

\end{otherlanguage}

\paragraph{}
Concerning $A_n$ in~\cite{highRankAlternating} the authors proved that for each rang $\ge 3$, there exists an abstract regular polytope that admits $A_n$ as an automorphism group for a given $n \ge 9$.

\paragraph{}
One of the question that arose is to know the maximum rank for an abstrat regular polytope on $A_n$. It was conjectured in 2012~\cite{A12PolytopesRank} that, the maximal rank of a polytope on $A_n$ is $\left \lfloor \frac{n-1}{2} \right \rfloor$ for $n \ge 12$. This conjecture was proven in 2017 in~\cite{highestRankOfAn}.

\paragraph{}
The goal of this paper is to understand why $A_{11}$ does not admit any abstract regular polytope of rank 4 or of rank 5 despite the fact that there are polytopes of rank 3 and of rank 6.

\paragraph{}
This work is structured in the following way. In Chapter~\ref{Preliminaries}, we discuss some fondamental notions of graphs, combinatorics, group theory and euclidian polytopes. Then we introduce the concept of abstract polytopes, permutation representation graphs and some well-known results about them.

\paragraph{}
In chapter~\ref{Properties}, we develop some additional properties about permutation representation graphs that are used in the next chapters. It is also the occasion to familiarize the reader with the manipulation of such graphs with some small and easy proofs.

\paragraph{}
In chapters~\ref{proof-5} and~\ref{proof-4}, we apply the results and the methods explained in chapter~\ref{Properties} to prove that there does not exist any polytope of rank 5 or 4 with automorphism group $A_{11}$.

\paragraph{}
The research in this work has been greatly helped by the software \textsc{Magma}~\cite{magma} which helped us to find guidelines and rapidly check hypotheses.

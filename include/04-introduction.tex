\chapter*{Introduction}
\addcontentsline{toc}{chapter}{Introduction}

\begin{otherlanguage}{french}

\paragraph{}
Le début de l'étude des polyèdres réguliers remonte aux Grecs. On attribue leur découverte à Pythagore (570-476). En particulier les  polytopes réguliers de dimension trois ont été classifiés. Cette classification est attribuée à Platon, d'où le nom de \textit{solides platoniciens}.

\paragraph{}
Pendant presque deux millénaires, peu de progrès ont été effectués à l'exception notable de la classification des polyèdres étoilés de dimension trois réalisée par Kepler au début du 17ème siècle. Il a fallu attendre le 19ème siècle pour que les scientifiques s'intéressent aux polyèdres de dimensions supérieures à trois.

\paragraph{}
De nombreux résultats ont à partir de ce moment été trouvés. Ces résultats étaient valables pour des polytopes de dimensions quelconques. Cependant ces résultats s'appliquent aussi à un ensemble plus grand d'objets qui ne sont pas tous des polytopes. Ces objets ont été nommés \textit{polytopes abstraits}.

\paragraph{}
L'étude de ces polytopes abstraits s'est rapidement concentrée sur des familles de polytopes avec certaines propriétés de symétrie: les polytopes abstraits réguliers ainsi que les polytopes abstraits chiraux. Ce haut degré de symétrie rend ces polytopes abstraits intéressants à étudier au moyen des groupes de permutations.

\paragraph{}
La question a fini par s'inverser et il est devenu plus intéressant de trouver tous les polytopes admettant un certain groupe d'automorphisme. La recherche s'est assez vite concentrée sur les groupes presque simples.

\paragraph{}
Au milieu du 20ème siècle, une correspondance a été établie entre les polytopes abstraits réguliers et ce qu'on a appelé des \textit{string C-group representations}. Ceci a simplifié l'étude des polytopes abstraits. En effet, ces string C-group representations sont des représentations du groupe d'automorphismes d'un polytope abstrait qui permettent en outre de définir complètement ce polytope abstrait. Ceci a permis d'utiliser des résultats de la théorie des groupes de permutations dans l'étude des polytopes abstraits.

\paragraph{}
Un atlas de tous les polytopes admettant de petit groupes presque simples a été repris dans~\cite{atlasPolytopes}.

\paragraph{}
Concernant les grands groupes, la plupart des recherches se sont concentrées sur $S_n$ et $A_n$. Concernant $S_n$, il a été prouvé dans~\cite{highRankSym} qu'il existe exactement un polytope de rang $n-1$ qui admet $S_n$ comme groupe d'automorphismes pour $n \ge 5$. De même il a été prouvé qu'il existe exactement un polytope abstrait de rang $n-2$ qui admet comme groupe de symétrie $S_n$ pour $n \ge 7$. Plus récemment, il a été prouvé dans~\cite{leemansTransactions} qu'il existe exactement 7 polytopes abstraits de rang $n-3$ qui ont comme groupe d'automorphismes $S_n$ pour $n \ge 9$. Il est actuellement conjecturé que cette propriété est vraie pour 9 polytopes de rang $n-3$ pour $n \ge 11$. C'est une question actuellement ouverte.

\paragraph{}
Concernant $A_n$, dans~\cite{highRankAlternating} les auteurs prouvent que pour chaque rang $\ge 3$, il existe un polytope abstrait qui admet comme groupe d'automorphismes $A_n$ pour un certain $n \ge 9$.

\paragraph{}
Une des questions qui en a découlé est de connaître le rang maximal pour un polytope qui admet comme groupe d'automorphismes $A_n$. Il a été conjecturé en 2012 dans~\cite{A12PolytopesRank} que, le rang maximale d'un polytope admettant $A_n$ comme groupe d'automorphismes est $\left\lfloor\frac{n-1}{2}\right\rfloor$ pour $n \ge 12$. Cette conjecture a été prouvée en 2017 dans~\cite{highestRankOfAn}.

\paragraph{}
Durant cette recherche les auteurs ont calculé les polytopes de $A_{11}$. Ils ont constaté que les polytopes de ce groupe avaient soit une rang de trois, soit de six. Ceci a attiré leur attention car, actuellement, il s'agit du seul groupe connu dont l'ensemble des rangs acceptables pour un polytope n'est pas un intervalle.

\paragraph{}
L'objectif du présent travail est de comprendre pourquoi $A_{11}$ n'admet aucun polytope de rang 4 ou de rang 5 malgré le fait qu'il existe des polytopes de rang 3 et de rang 6.

\end{otherlanguage}

\paragraph{}
This work is structured in the following way. In Chapter~\ref{Preliminaries}, we discuss some fondamental notions of graphs, combinatorics, group theory and euclidian polytopes. Then we introduce the concept of abstract polytopes, permutation representation graphs and some well-known results about them.

\paragraph{}
In chapter~\ref{Properties}, we develop some additional properties about permutation representation graphs that are used in the next chapters. It is also the occasion to familiarize the reader with the manipulation of such graphs with some small and easy proofs.

\paragraph{}
In chapters~\ref{proof-5} and~\ref{proof-4}, we apply the results and the methods explained in chapter~\ref{Properties} to prove that there does not exist any polytopes of rank 5 or 4.

\paragraph{}
The research in this work has been greatly help by the software \textsc{Magma}~\cite{magma} which helped us to find guidelines and rapidly check hypothesis.

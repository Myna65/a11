\chapter*{Introduction}
\addcontentsline{toc}{chapter}{Introduction}

\begin{otherlanguage}{french}

\paragraph{}
Le début de l'étude des polyhèdres réguliers remonte aux grecs. On attribue leur découverte à Pythagore (570-476)

Historique ->



\paragraph{}
Parler des polyèdres réguliers et des grecs

\paragraph{}
Cette notion de polyèdre régulier fut le s

\paragraph{}
Les polytopes abstraits on donc été introduit par en comme une généralisation des polytopes. Ces polytopes abstrait

\paragraph{}
Parler des polytopes abstraits

\paragraph{}
L'étude de ces polytopes abstraits s'est rapidement concentrée sur des familles de polytopes avec certaines propriétés de symétrie. Les polytopes abstraits réguliers ainsi que les polytopes abstraits chiraux. Ces polytopes abstraits étant très symmétriques, leurs groupes de symmétrie présente un certain intérêt à être étudié d'un point de vue des groupes de permutations.

\paragraph{}
La question a fini par s'inverser et il est devenu intéressant de trouver tous les polytopes admettant un certain groupe comme groupe d'automorphisme. La recherche s'est assez vite concentrée sur les groupes presques simples et en particulier sur $S_n$ et $A_n$.

\paragraph{}
Une correspondance a été trouvée entre les polyoptes abstraits réguliers et ce qu'on apelle des string C-group representation.

\paragraph{}
Cette correspondance a été appronfondie particulièrement pour les groupes $A_n$. En effet, dans~\cite{} les auteurs prouvent que pour chaque rang $\ge 3$, il existe un polytope abstrait qui admet comme group d'automorphisme $A_n$ avec $n \ge 9$. Une des questions qui est ensuite arrivée est de savoir quel est le rang maximal pour un polytope qui admet comme groupe d'automorphism $A_n$.

\paragraph{}
Il a été conjecturé en 2012 que, pour un rang donné, 

\paragraph{}
Parler des polytopes avec beaucoup de symétries (réguliers et chiraux)

Groupe presque simples sur des polytopes

\paragraph{}
Parler des avancées récentes

\paragraph{}
Et de comment ceci vient les compléter

\end{otherlanguage}

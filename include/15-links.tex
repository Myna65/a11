\subsection{Permutation representation graphs}

\paragraph{}

\begin{definition}[Permutation representation graph]
  \index{permutation representation graph}
  Let $\mathcal P$ be a regular polytope of dimension $d$ and let $\pi$ be an embedding of the automorphism group of $\Gamma$ into $S_n$ for some $n$. The permutation representation graph of $\mathcal P$ is a labeled multigraph without loops with vertices ${1 \dots n}$, labels ${1 \dots d}$ such that there is an edge of label $k$ between $i$ and $j$ if $(\pi \rho_k)i = j$ and $i \neq j$. The labels of those graphs are called indices.
\end{definition}

\paragraph{}
Permutation representation graphs were introduced by Pellicier in~\cite{cprGraph} under the name "CPR graph". He requires as an additional condition that the group satisfies the intersection property. But given the graph, it is not easy to check whether it is a C-group or not. Therefore we chose to remove the intersection condition and call it "permutation representation graph".

\paragraph{}
Here is one example of a permutation representation graph:

\begin{figure}[H]
  \begin{center}
    \begin{tikzpicture}[scale=.8]

      \begin{scope}[every node/.style={circle,draw, transform shape}]
        \node (1)  at (6,0)  {};
        \node (2)  at (6,2)  {};
        \node (3)  at (4,2)  {};
        \node (4)  at (4,0)  {};
        \node (5)  at (4,-2) {};
        \node (6)  at (2,-2) {};
        \node (7)  at (2,0)  {};
        \node (8)  at (2,2)  {};
        \node (9)  at (12,0) {};
        \node (10) at (10,0) {};
        \node (11) at (8,0) {};
      \end{scope}

      \begin{scope}[every node/.style={fill=white, transform shape}]

        \begin{scope}[every edge/.style={draw}]
          \path (1)  edge node {$0$} (2);
          \path (3)  edge node {$0$} (4);
          \path (5)  edge[bend right=50] node {$0$} (6);
          \path (7)  edge node {$0$} (8);
          \path (1)  edge node {$1$} (11);
          \path (3)  edge[bend right=30] node {$1$} (8);
          \path (4)  edge node {$1$} (5);
          \path (6)  edge node {$1$} (7);
          \path (1)  edge node {$2$} (4);
          \path (2)  edge node {$2$} (3);
          \path (5)  edge node {$2$} (6);
          \path (10) edge node {$2$} (11);
          \path (3)  edge[bend left=30] node {$3$} (8);
          \path (4)  edge node {$3$} (7);
          \path (5)  edge[bend left=50] node {$3$} (6);
          \path (9)  edge node {$3$} (10);
        \end{scope}
      \end{scope}

    \end{tikzpicture}
    \caption{}
  \end{center}
\end{figure}

\paragraph{}
We denote by $\Gamma_{\rho_i, \rho_j}$ the subgraph where only the $\rho_i$ and $\rho_j$ edges are kept.

\begin{property}
  \label{intersection-patterns}
  Let $\Gamma = \langle \rho_0, \dots, \rho_n \rangle$ be a permutation representation graph of a regular polytope $\mathcal P$, and let $|i - j| \ge 2$. Then every connected component of $\Gamma_{\rho_i,\rho_j}$ is either a single vertex, a single edge, a double edge or an alternating square.
\end{property}

For the proof see the proposition 3.5 of~\cite{cprGraph}.

\begin{property}
  \label{generators-not-free}
  In an indexed group representation, if a generator can be written as a product of the other generators then it does not satisfy the intersection property and thus is not a C-group representation.
\end{property}

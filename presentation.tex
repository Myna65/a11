\documentclass[10pt]{beamer}

\usepackage[utf8]{inputenc}
\usepackage[french]{babel}

\usetheme[progressbar=frametitle]{metropolis}
\usepackage{appendixnumberbeamer}

\usepackage{booktabs}
\usepackage[scale=2]{ccicons}

\usepackage{pgfplots}
\usepgfplotslibrary{dateplot}

\usepackage{xspace}
\newcommand{\themename}{\textbf{\textsc{metropolis}}\xspace}

\usepackage{tikz}
\usepackage{svg}

\usepackage{amsthm}
\usepackage{amsfonts}
\usepackage{mathrsfs}

\usepackage{thmtools}

\usetikzlibrary{graphs}
\usetikzlibrary{graphs.standard}

\let\definition\relax
\let\theorem\relax
\newtheorem{definition}{Définition}[section]
\newtheorem{proposition}[definition]{Proposition}
\newtheorem{theorem}[definition]{Théorem}

\title{Abstract polytopes on $A_{11}$}
%\subtitle{Graphes de Moore}
% \date{\today}
\date{}
\author{Nathan Meynaert}
% \titlegraphic{\hfill\includegraphics[height=1.5cm]{logo.pdf}}

\begin{document}

\maketitle

\begin{frame}{Table des matières}
  \setbeamertemplate{section in toc}[sections numbered]
  \tableofcontents[hideallsubsections]
\end{frame}

\section{Polytopes abstraits}

\begin{frame}{Définition}

  \begin{definition}
    Un \textit{polytope abstrait} est un ensemble partiellement ordonné, gradué,  dont les éléments sont appelés \textit{faces} et sont ordonnés par l'inclusion $\subseteq$ tels que
    \begin{itemize}
      \item \textbf{(P1)} $\mathcal P$ contient deux faces impropres: $F_{-1}$ et $F_d$.
      \item \textbf{(P2)} Les drapeaux de $\mathcal P$ contiennent $d+2$ faces (en incluant les deux faces impropres)    \item \textbf{(P3)} $\mathcal P$ est strictement connecté.
      \item \textbf{(P4)} $\mathcal P$ satisfait la propriété du diamant.
    \end{itemize}
  \end{definition}

\end{frame}

\begin{frame}{Strictement connecté}
\end{frame}

\begin{frame}{4-transposition}
\end{frame}

\section{Permutation representation graphs}

\begin{frame}{Définition}
  Un permutation representation graph est un graphe représentant un sggi. Le sggi est composé de générateurs $\rho_0, \dots, \rho_{r-1}$. Ce graphe est un multigraphe avec labels dont les sommets sont les points du string C-group. Deux points $\alpha$ et $\beta$ sont reliés par une arête avec un label $i$ ssi $\alpha \rho_i = \beta$. Par convention, le graphe ne contient pas de boucle.


  \pause

  \begin{figure}[H]
    \begin{center}
      \begin{tikzpicture}[scale=.8]

        \begin{scope}[every node/.style={circle,draw, transform shape}]
          \node (1)  at (6,0)  {};
          \node (2)  at (6,2)  {};
          \node (3)  at (4,2)  {};
          \node (4)  at (4,0)  {};
          \node (5)  at (4,-2) {};
          \node (6)  at (2,-2) {};
          \node (7)  at (2,0)  {};
          \node (8)  at (2,2)  {};
          \node (9)  at (12,0) {};
          \node (10) at (10,0) {};
          \node (11) at (8,0) {};
        \end{scope}

        \begin{scope}[every node/.style={fill=white, transform shape}]

          \begin{scope}[every edge/.style={draw}]
            \path (1)  edge node {$0$} (2);
            \path (3)  edge node {$0$} (4);
            \path (5)  edge[bend right=50] node {$0$} (6);
            \path (7)  edge node {$0$} (8);
            \path (1)  edge node {$1$} (11);
            \path (3)  edge[bend right=30] node {$1$} (8);
            \path (4)  edge node {$1$} (5);
            \path (6)  edge node {$1$} (7);
            \path (1)  edge node {$2$} (4);
            \path (2)  edge node {$2$} (3);
            \path (5)  edge node {$2$} (6);
            \path (10) edge node {$2$} (11);
            \path (3)  edge[bend left=30] node {$3$} (8);
            \path (4)  edge node {$3$} (7);
            \path (5)  edge[bend left=50] node {$3$} (6);
            \path (9)  edge node {$3$} (10);
          \end{scope}
        \end{scope}

      \end{tikzpicture}
    \end{center}
  \end{figure}
\end{frame}

\begin{frame}{Motifs possibles}
  %Les générateurs non consecutifs d'un sggi doivent commuter, ceci a des conséquences sur le graphe.

  \begin{theorem}
    Dans un permutation representation graph, le graphe obtenu en ne conservant que les arêtes de deux générateurs non-consécutifs peuvent uniquement former les motifs suivants:
    \begin{itemize}
      \item Un point fixe
      \item Une arête simple
      \item Une arête double
      \item Un carré alterné
    \end{itemize}
  \end{theorem}

  \pause

  \begin{figure}[H]
    \begin{center}
      \begin{tikzpicture}[scale=.55]

        \begin{scope}[every node/.style={circle,draw, transform shape}]
          \node (1)  at (0,0)  {};
          \node (2)  at (2,0)  {};
          \node (3)  at (4,0)  {};
          \node (4)  at (6,0)  {};
          \node (5)  at (8,0) {};
          \node (6)  at (10,0) {};
          \node (7)  at (12,0)  {};
          \node (8)  at (14,0)  {};
          \node (9)  at (14,2) {};
          \node (10) at (16,0) {};
          \node (11) at (16,2) {};
        \end{scope}

        \begin{scope}[every node/.style={fill=white, transform shape}]

          \begin{scope}[every edge/.style={draw}]
            \path (2)  edge node {$i$} (3);
            \path (6)  edge[bend left=30] node {$i$} (7);
            \path (8)  edge node {$i$} (10);
            \path (9)  edge node {$i$} (11);
            \path (4)  edge node {$j$} (5);
            \path (6)  edge[bend right=30] node {$j$} (7);
            \path (8)  edge node {$j$} (9);
            \path (10) edge node {$j$} (11);
          \end{scope}
        \end{scope}

      \end{tikzpicture}
    \end{center}
  \end{figure}

\end{frame}

\begin{frame}{Exemple (1)}
  Un exemple:

  \begin{figure}[H]
    \begin{center}
      \begin{tikzpicture}[scale=.8]

        \begin{scope}[every node/.style={circle,draw, transform shape}]
          \node (1)  at (6,0)  {};
          \node (2)  at (6,2)  {};
          \node (3)  at (4,2)  {};
          \node (4)  at (4,0)  {};
          \node (5)  at (4,-2) {};
          \node (6)  at (2,-2) {};
          \node (7)  at (2,0)  {};
          \node (8)  at (2,2)  {};
          \node (9)  at (12,0) {};
          \node (10) at (10,0) {};
          \node (11) at (8,0) {};
        \end{scope}

        \begin{scope}[every node/.style={fill=white, transform shape}]

          \begin{scope}[every edge/.style={draw}]
            \path (1)  edge node {$0$} (2);
            \path (3)  edge node {$0$} (4);
            \path (5)  edge[bend right=50] node {$0$} (6);
            \path (7)  edge node {$0$} (8);
            \path (1)  edge node {$1$} (11);
            \path (3)  edge[bend right=30] node {$1$} (8);
            \path (4)  edge node {$1$} (5);
            \path (6)  edge node {$1$} (7);
            \path (1)  edge node {$2$} (4);
            \path (2)  edge node {$2$} (3);
            \path (5)  edge node {$2$} (6);
            \path (10) edge node {$2$} (11);
            \path (3)  edge[bend left=30] node {$3$} (8);
            \path (4)  edge node {$3$} (7);
            \path (5)  edge[bend left=50] node {$3$} (6);
            \path (9)  edge node {$3$} (10);
          \end{scope}
        \end{scope}

      \end{tikzpicture}
    \end{center}
  \end{figure}
\end{frame}

\begin{frame}{Exemple (2)}
  Un second exemple plus compliqué:

  \begin{figure}[H]
    \begin{center}
      \begin{tikzpicture}[scale=.8]

        \begin{scope}[every node/.style={circle,draw, transform shape}]
          \node (1)  at (0,2)  {h};
          \node (2)  at (0,0)  {g};
          \node (3)  at (0,-2) {f};
          \node (4)  at (-2,2)  {e};
          \node (5)  at (-2,0)  {d};
          \node (6)  at (-2,-2) {c};
          \node (7)  at (-4,2)  {b};
          \node (8)  at (-4,0)  {a};
          \node (9)  at (-2,4)  {i};
          \node (10) at (0,4)   {j};
          \node (11) at (2,0)   {k};
          \node (12) at (2,2)   {l};
        \end{scope}

        \begin{scope}[every node/.style={fill=white, transform shape}]

          \node (t1) at (-5,1)  {$i$};
          \node (t2) at (3,1)   {$i$};
          \node (t3) at (-1,-3) {$i+1$};
          \node (t4) at (-1,5)  {$i+1$};

          \begin{scope}[every edge/.style={draw}]
            \path (2)  edge node {$i+2$} (3);
            \path (5)  edge node {$i+2$} (6);
            \path (9)  edge node {$i+2$} (4);
            \path (10) edge node {$i+2$} (1);
            \path (1)  edge node {$i$} (2);
            \path (4)  edge node {$i$} (5);
            \path (7)  edge node {$i$} (8);
            \path (11) edge node {$i$} (12);
            \path (9)  edge[bend right=45] (t1);
            \path (t1) edge[bend right=45] (6);
            \path (10) edge[bend left=45] (t2);
            \path (t2) edge[bend left=45] (3);
            \path (12) edge[bend right=45] (t4);
            \path (t4) edge[bend right=45] (7);
            \path (11) edge[bend left=45] (t3);
            \path (t3) edge[bend left=45] (8);
            \path (1)  edge node {$i+1$} (4);
            \path (2)  edge node {$i+1$} (5);
            \path (3)  edge node {$i+1$} (6);
            \path (4)  edge node {$i+3$} (7);
            \path (5)  edge node {$i+3$} (8);
            \path (1)  edge node {$i+3$} (12);
            \path (2)  edge node {$i+3$} (11);
          \end{scope}
        \end{scope}

      \end{tikzpicture}
    \end{center}
  \end{figure}
\end{frame}

\section{Nombre minimal de 4-transpositions}

\begin{frame}{Rang 5}
  \begin{theorem}
    Tout permutation representation graph de rang 5 de $A_{11}$ admet au moins une 4-transposition.
  \end{theorem}
\end{frame}

\begin{frame}{Preuve (1)}
    Supposons qu'il n'y ait aucune 4-transposition alors tous les générateurs doivent être des 2-transpositions et donc il y a 10 arêtes pour lier 11 points. Le graphe doit donc être un arbre.

    Cependant aucun sommet ne peut être relié par 3 arêtes.

    \begin{figure}[H]
      \begin{center}
        \begin{tikzpicture}[scale=.8]

          \begin{scope}[every node/.style={circle,draw, transform shape}]
            \node (1)  at (0,0)  {};
            \node (2)  at (2,0)  {};
            \node (3)  at (4,0)  {};
            \node (4)  at (2,2)  {};
          \end{scope}

          \begin{scope}[every node/.style={fill=white, transform shape}]
            \begin{scope}[every edge/.style={draw}]
              \path (1)  edge node {$i$} (2);
              \path (2)  edge node {$j$} (3);
              \path (2)  edge node {$k$} (4);
            \end{scope}
          \end{scope}

        \end{tikzpicture}
      \end{center}
    \end{figure}

    Supposons que $i$ et $k$ soient non-consécutifs alors les deux arêtes $i$ et $k$ de ce graphe doivent être dans un motif valide. Le seul possible est le carré alterné mais alors le graphe n'est pas un arbre.

    Donc le graphe doit être une chaîne.
\end{frame}

\begin{frame}{Preuve (2)}

  Dans une chaîne, les indices des arêtes adjacentes doivent être consécutifs. En particulier les arêtes $\rho_0$ ne peuvent être adjacentes qu'à des arêtes $\rho_1$.

  Vu qu'il y a deux arêtes de chaque type, une arête $\rho_0$ doit être à l'extrémité du graphe. Si une arête $\rho_0$ ne se trouve pas à l'extrémité du graphe, alors elle doit être entourée de deux arêtes $\rho_1$.

  \begin{figure}[H]
    \begin{center}
      \begin{tikzpicture}[scale=.8]

        \begin{scope}[every node/.style={circle,draw, transform shape}]
          \node (1)  at (0,0)  {};
          \node (2)  at (2,0)  {};
          \node (3)  at (4,0)  {};
          \node (4)  at (6,0)  {};
        \end{scope}

        \begin{scope}[every node/.style={fill=white, transform shape}]
          \begin{scope}[every edge/.style={draw}]
            \path (1)  edge node {$1$} (2);
            \path (2)  edge node {$0$} (3);
            \path (3)  edge node {$1$} (4);
          \end{scope}
        \end{scope}

      \end{tikzpicture}
    \end{center}
  \end{figure}

  \pause

  La seconde arête $\rho_0$ doit être adjacente à une de ces arêtes $\rho_1$ mais ne peut avoir d'arête adjacente à son autre extrémité.

  \begin{figure}[H]
    \begin{center}
      \begin{tikzpicture}[scale=.8]

        \begin{scope}[every node/.style={circle,draw, transform shape}]
          \node (0)  at (-2,0)  {};
          \node (1)  at (0,0)  {};
          \node (2)  at (2,0)  {};
          \node (3)  at (4,0)  {};
          \node (4)  at (6,0)  {};
        \end{scope}

        \begin{scope}[every node/.style={fill=white, transform shape}]
          \begin{scope}[every edge/.style={draw}]
            \path (0)  edge node {$0$} (1);
            \path (1)  edge node {$1$} (2);
            \path (2)  edge node {$0$} (3);
            \path (3)  edge node {$1$} (4);
          \end{scope}
        \end{scope}

      \end{tikzpicture}
    \end{center}
  \end{figure}



\end{frame}

\begin{frame}{Preuve (3)}
  Par dualité, la même conclusion s'applique à $\rho_4$, la chaîne doit donc commencer par ces motifs à chaque extrémité.

  \begin{figure}[H]
    \begin{center}
      \begin{tikzpicture}[scale=.5]

        \begin{scope}[every node/.style={circle,draw, transform shape}]
          \node (1)  at (0,0)  {};
          \node (2)  at (2,0)  {};
          \node (3)  at (4,0)  {};
          \node (4)  at (6,0)  {};
          \node (5)  at (8,0)  {};
          \node (7)  at (11,0) {};
          \node (8)  at (13,0) {};
          \node (9)  at (15,0) {};
          \node (10) at (17,0) {};
          \node (11) at (19,0) {};
        \end{scope}

        \node (5b) at (9,0) {};
        \node (7b) at (10,0) {};

        \begin{scope}[every node/.style={fill=white, transform shape}]

          \begin{scope}[every edge/.style={draw}]
            \path (1)  edge node {$0$} (2);
            \path (2)  edge node {$1$} (3);
            \path (3)  edge node {$0$} (4);
            \path (4)  edge node {$1$} (5);
            \path (5)  edge[style={dotted}] (5b);
            \path (7)  edge[style={dotted}] (7b);
            \path (7)  edge node {$3$} (8);
            \path (8)  edge node {$4$} (9);
            \path (9)  edge node {$3$} (10);
            \path (10) edge node {$4$} (11);
          \end{scope}
        \end{scope}
      \end{tikzpicture}
    \end{center}
  \end{figure}

  \pause

  Il ne reste que des arêtes $\rho_2$ à placer.

  \begin{figure}[H]
    \begin{center}
      \begin{tikzpicture}[scale=.5]

        \begin{scope}[every node/.style={circle,draw, transform shape}]
          \node (1)  at (0,0)  {};
          \node (2)  at (2,0)  {};
          \node (3)  at (4,0)  {};
          \node (4)  at (6,0)  {};
          \node (5)  at (8,0)  {};
          \node (6)  at (10,0)  {};
          \node (7)  at (12,0) {};
          \node (8)  at (14,0) {};
          \node (9)  at (16,0) {};
          \node (10) at (18,0) {};
          \node (11) at (20,0) {};
        \end{scope}

        \begin{scope}[every node/.style={fill=white, transform shape}]

          \begin{scope}[every edge/.style={draw}]
            \path (1)  edge node {$0$} (2);
            \path (2)  edge node {$1$} (3);
            \path (3)  edge node {$0$} (4);
            \path (4)  edge node {$1$} (5);
            \path (7)  edge node {$3$} (8);
            \path (8)  edge node {$4$} (9);
            \path (9)  edge node {$3$} (10);
            \path (10) edge node {$4$} (11);
          \end{scope}
        \end{scope}
      \end{tikzpicture}
    \end{center}
  \end{figure}

  Ce qui est clairement impossible et donc aucune chaîne ne peut être construite.

\end{frame}

\begin{frame}{Rang 4}
  \begin{theorem}
    Tout permutation representation graph de rang 4 de $A_{11}$ admet au moins deux 4-transpositions.
  \end{theorem}
\end{frame}

\section{Méthode}

\begin{frame}{Méthode}
  Une méthode permettant de construire tous les graphes possibles est la suivante:

  \begin{enumerate}
    \item Partir d'une involution de base
    \item À partir d'une deuxième involution, sachant qu'il existe uniquement quatre motifs valides, lister toutes les possibilités.
    \item Étendre le graphe afin qu'il devienne connexe. Des arêtes supplémentaire peuvent être placées afin de satisfaire diverses contraintes.
    \item Ajouter les arêtes restantes.
  \end{enumerate}
\end{frame}

\section{Ni $\rho_0$ ni $\rho_4$ ne peut être une 4-transposition}

\begin{frame}{Intersection}
  Supposons que $\rho_0$ soit une 4-transposition, alors il permute 8 points. Cependant $\rho_4$ permute au moins 4 points et donc il existe un point qui est permuté par $\rho_0$ et $\rho_4$.

  Ce point doit faire partie d'un des 4 motifs. Les deux seuls possibles sont le carré alterné et l'arête double.

  Les motifs de départ sont donc les suivants:

  \begin{figure}[H]
    \begin{center}
      \begin{tikzpicture}[scale=.55]

        \begin{scope}[every node/.style={circle,draw, transform shape}]
          \node (6)  at (10,0) {};
          \node (7)  at (12,0) {};
          \node (8)  at (14,0) {};
          \node (9)  at (14,2) {};
          \node (10) at (16,0) {};
          \node (11) at (16,2) {};
        \end{scope}

        \begin{scope}[every node/.style={fill=white, transform shape}]

          \begin{scope}[every edge/.style={draw}]
            \path (6)  edge[bend left=30] node {$0$} (7);
            \path (8)  edge node {$0$} (10);
            \path (9)  edge node {$0$} (11);
            \path (6)  edge[bend right=30] node {$4$} (7);
            \path (8)  edge node {$4$} (9);
            \path (10) edge node {$4$} (11);
          \end{scope}
        \end{scope}

      \end{tikzpicture}
    \end{center}
  \end{figure}

\end{frame}

\begin{frame}{Carré alterné (1)}
  Partons du carré alterné

  \begin{figure}[H]
    \begin{center}
      \begin{tikzpicture}[scale=.55]

        \begin{scope}[every node/.style={circle,draw, transform shape}]
          \node (1)  at (0,0) {};
          \node (2)  at (0,2) {};
          \node (3)  at (2,0) {};
          \node (4)  at (2,2) {};
        \end{scope}

        \begin{scope}[every node/.style={fill=white, transform shape}]

          \begin{scope}[every edge/.style={draw}]
            \path (1)  edge node {$0$} (3);
            \path (2)  edge node {$0$} (4);
            \path (1)  edge node {$4$} (2);
            \path (3)  edge node {$4$} (4);
          \end{scope}
        \end{scope}

      \end{tikzpicture}
    \end{center}
  \end{figure}

  \pause

  Ce carré alterné ne peut être adjacent à une arête simple.

  Il doit donc être adjacent à un autre carré alterné. Les possibilités sont un carré avec $\rho_0$ et $\rho_3$ ou un carré avec $\rho_1$ et $\rho_4$. Le second cas peut être ramené au premier par dualité.

  \pause

  \begin{figure}[H]
    \begin{center}
      \begin{tikzpicture}[scale=.55]

        \begin{scope}[every node/.style={circle,draw, transform shape}]
          \node (1)  at (0,0) {};
          \node (2)  at (0,2) {};
          \node (3)  at (2,0) {};
          \node (4)  at (2,2) {};
          \node (5)  at (4,0) {};
          \node (6)  at (4,2) {};
        \end{scope}

        \begin{scope}[every node/.style={fill=white, transform shape}]

          \begin{scope}[every edge/.style={draw}]
            \path (1)  edge node {$0$} (2);
            \path (3)  edge node {$0$} (4);
            \path (5)  edge node {$0$} (6);
            \path (5)  edge node {$3$} (3);
            \path (6)  edge node {$3$} (4);
            \path (1)  edge node {$4$} (3);
            \path (2)  edge node {$4$} (4);
          \end{scope}
        \end{scope}

      \end{tikzpicture}
    \end{center}
  \end{figure}

\end{frame}

\begin{frame}{Carré alterné (2)}
  \begin{figure}[H]
    \begin{center}
      \begin{tikzpicture}[scale=.55]

        \begin{scope}[every node/.style={circle,draw, transform shape}]
          \node (1)  at (0,0) {};
          \node (2)  at (0,2) {};
          \node (3)  at (2,0) {};
          \node (4)  at (2,2) {};
          \node (5)  at (4,0) {};
          \node (6)  at (4,2) {};
        \end{scope}

        \begin{scope}[every node/.style={fill=white, transform shape}]

          \begin{scope}[every edge/.style={draw}]
            \path (1)  edge node {$0$} (2);
            \path (3)  edge node {$0$} (4);
            \path (5)  edge node {$0$} (6);
            \path (5)  edge node {$3$} (3);
            \path (6)  edge node {$3$} (4);
            \path (1)  edge node {$4$} (3);
            \path (2)  edge node {$4$} (4);
          \end{scope}
        \end{scope}

      \end{tikzpicture}
    \end{center}
  \end{figure}

  Un troisième carré alterné doit être placé, ce carré doit être composé de $\rho_0$ et $\rho_2$.

  \pause

  \begin{figure}[H]
    \begin{center}
      \begin{tikzpicture}[scale=.55]

        \begin{scope}[every node/.style={circle,draw, transform shape}]
          \node (1)  at (0,0) {};
          \node (2)  at (0,2) {};
          \node (3)  at (2,0) {};
          \node (4)  at (2,2) {};
          \node (5)  at (4,0) {};
          \node (6)  at (4,2) {};
          \node (7)  at (6,0) {};
          \node (8)  at (6,2) {};
        \end{scope}

        \begin{scope}[every node/.style={fill=white, transform shape}]

          \begin{scope}[every edge/.style={draw}]
            \path (1)  edge node {$0$} (2);
            \path (3)  edge node {$0$} (4);
            \path (5)  edge node {$0$} (6);
            \path (7)  edge node {$0$} (8);
            \path (5)  edge node {$2$} (7);
            \path (6)  edge node {$2$} (8);
            \path (5)  edge node {$3$} (3);
            \path (6)  edge node {$3$} (4);
            \path (1)  edge node {$4$} (3);
            \path (2)  edge node {$4$} (4);
          \end{scope}
        \end{scope}

      \end{tikzpicture}
    \end{center}
  \end{figure}

\end{frame}

\begin{frame}{Carré alterné (3)}
  \begin{figure}[H]
    \begin{center}
      \begin{tikzpicture}[scale=.55]

        \begin{scope}[every node/.style={circle,draw, transform shape}]
          \node (1)  at (0,0) {};
          \node (2)  at (0,2) {};
          \node (3)  at (2,0) {};
          \node (4)  at (2,2) {};
          \node (5)  at (4,0) {};
          \node (6)  at (4,2) {};
          \node (7)  at (6,0) {};
          \node (8)  at (6,2) {};
        \end{scope}

        \begin{scope}[every node/.style={fill=white, transform shape}]

          \begin{scope}[every edge/.style={draw}]
            \path (1)  edge node {$0$} (2);
            \path (3)  edge node {$0$} (4);
            \path (5)  edge node {$0$} (6);
            \path (7)  edge node {$0$} (8);
            \path (5)  edge node {$2$} (7);
            \path (6)  edge node {$2$} (8);
            \path (5)  edge node {$3$} (3);
            \path (6)  edge node {$3$} (4);
            \path (1)  edge node {$4$} (3);
            \path (2)  edge node {$4$} (4);
          \end{scope}
        \end{scope}

      \end{tikzpicture}
    \end{center}
  \end{figure}

  Maintenant il est possible de connecter une arête simple, celle-ci doit être $\rho_1$.

  \pause

  \begin{figure}[H]
    \begin{center}
      \begin{tikzpicture}[scale=.55]

        \begin{scope}[every node/.style={circle,draw, transform shape}]
          \node (1)  at (0,0) {};
          \node (2)  at (0,2) {};
          \node (3)  at (2,0) {};
          \node (4)  at (2,2) {};
          \node (5)  at (4,0) {};
          \node (6)  at (4,2) {};
          \node (7)  at (6,0) {};
          \node (8)  at (6,2) {};
          \node (9)  at (8,0) {};
        \end{scope}

        \begin{scope}[every node/.style={fill=white, transform shape}]

          \begin{scope}[every edge/.style={draw}]
            \path (1)  edge node {$0$} (2);
            \path (3)  edge node {$0$} (4);
            \path (5)  edge node {$0$} (6);
            \path (7)  edge node {$0$} (8);
            \path (7)  edge node {$1$} (9);
            \path (5)  edge node {$2$} (7);
            \path (6)  edge node {$2$} (8);
            \path (5)  edge node {$3$} (3);
            \path (6)  edge node {$3$} (4);
            \path (1)  edge node {$4$} (3);
            \path (2)  edge node {$4$} (4);
          \end{scope}
        \end{scope}

      \end{tikzpicture}
    \end{center}
  \end{figure}

  Maintenant une arête $\rho_2$ doit être placée.

  \begin{figure}[H]
    \begin{center}
      \begin{tikzpicture}[scale=.55]

        \begin{scope}[every node/.style={circle,draw, transform shape}]
          \node (1)  at (0,0) {};
          \node (2)  at (0,2) {};
          \node (3)  at (2,0) {};
          \node (4)  at (2,2) {};
          \node (5)  at (4,0) {};
          \node (6)  at (4,2) {};
          \node (7)  at (6,0) {};
          \node (8)  at (6,2) {};
          \node (9)  at (8,0) {};
          \node (10) at (10,0) {};
        \end{scope}

        \begin{scope}[every node/.style={fill=white, transform shape}]

          \begin{scope}[every edge/.style={draw}]
            \path (1)  edge node {$0$} (2);
            \path (3)  edge node {$0$} (4);
            \path (5)  edge node {$0$} (6);
            \path (7)  edge node {$0$} (8);
            \path (7)  edge node {$1$} (9);
            \path (5)  edge node {$2$} (7);
            \path (6)  edge node {$2$} (8);
            \path (9)  edge node {$2$} (10);
            \path (5)  edge node {$3$} (3);
            \path (6)  edge node {$3$} (4);
            \path (1)  edge node {$4$} (3);
            \path (2)  edge node {$4$} (4);
          \end{scope}
        \end{scope}

      \end{tikzpicture}
    \end{center}
  \end{figure}
\end{frame}

\begin{frame}{Carré alterné (4)}
  \begin{figure}[H]
    \begin{center}
      \begin{tikzpicture}[scale=.55]

        \begin{scope}[every node/.style={circle,draw, transform shape}]
          \node (1)  at (0,0) {};
          \node (2)  at (0,2) {};
          \node (3)  at (2,0) {};
          \node (4)  at (2,2) {};
          \node (5)  at (4,0) {};
          \node (6)  at (4,2) {};
          \node (7)  at (6,0) {};
          \node (8)  at (6,2) {};
          \node (9)  at (8,0) {};
          \node (10) at (10,0) {};
        \end{scope}

        \begin{scope}[every node/.style={fill=white, transform shape}]

          \begin{scope}[every edge/.style={draw}]
            \path (1)  edge node {$0$} (2);
            \path (3)  edge node {$0$} (4);
            \path (5)  edge node {$0$} (6);
            \path (7)  edge node {$0$} (8);
            \path (7)  edge node {$1$} (9);
            \path (5)  edge node {$2$} (7);
            \path (6)  edge node {$2$} (8);
            \path (9)  edge node {$2$} (10);
            \path (5)  edge node {$3$} (3);
            \path (6)  edge node {$3$} (4);
            \path (1)  edge node {$4$} (3);
            \path (2)  edge node {$4$} (4);
          \end{scope}
        \end{scope}

      \end{tikzpicture}
    \end{center}
  \end{figure}

  Un point doit encore être relié mais le nombre total d'arêtes $\rho_2$ est pour le moment impair. Donc soit le dernier point doit être relié par une arête $\rho_2$ soit une arête double contenant $\rho_2$ doit être créée entre deux points déjà reliés.

  Cependant aucune de ces options n'est possible donc aucun permutation representation graph ne peut être trouvé.


\end{frame}

\section{Ni $\rho_1$ ni $\rho_3$ ne peut être une 4-transposition}

\begin{frame}{Graphes possibles}

  Après application de la méthode dans ce cas, nous trouvons deux familles de graphes. Voici un exemple pour chaque famille.

  \begin{figure}[H]
    \begin{center}
      \begin{tikzpicture}[scale=.55]

        \begin{scope}[every node/.style={circle,draw, transform shape}]
          \node (1)  at (0,2)  {};
          \node (2)  at (0,0)  {};
          \node (3)  at (2,2)  {};
          \node (4)  at (2,0)  {};
          \node (5)  at (4,2)  {};
          \node (6)  at (4,0)  {};
          \node (7)  at (6,0)  {};
          \node (8)  at (8,2)  {};
          \node (9)  at (8,0)  {};
          \node (10) at (10,2) {};
          \node (11) at (10,0) {};
        \end{scope}

        \begin{scope}[every node/.style={fill=white, transform shape}]

          \begin{scope}[every edge/.style={draw}]
            \path (9)  edge node {$0$} (11);
            \path (8)  edge[bend left=30] node {$0$} (10);
            \path (1)  edge[bend right=30] node {$1$} (2);
            \path (3)  edge node {$1$} (4);
            \path (5)  edge node {$1$} (6);
            \path (7)  edge node {$1$} (9);
            \path (1)  edge[bend left=30] node {$2$} (3);
            \path (6)  edge node {$2$} (7);
            \path (8)  edge node {$2$} (9);
            \path (10) edge node {$2$} (11);
            \path (1)  edge[bend left=30] node {$3$} (2);
            \path (3)  edge node {$3$} (5);
            \path (4)  edge node {$3$} (6);
            \path (8)  edge[bend right=30] node {$3$} (10);
            \path (1)  edge[bend right=30] node {$4$} (3);
            \path (2)  edge node {$4$} (4);
          \end{scope}
        \end{scope}

      \end{tikzpicture}
    \end{center}
  \end{figure}

  \begin{figure}[H]
    \begin{center}
      \begin{tikzpicture}[scale=.55]

        \begin{scope}[every node/.style={circle,draw, transform shape}]
          \node (1)  at (0,2)  {};
          \node (2)  at (0,0)  {};
          \node (3)  at (2,2)  {};
          \node (4)  at (2,0)  {};
          \node (5)  at (4,0)  {};
          \node (6)  at (6,0)  {};
          \node (7)  at (8,0)  {};
          \node (8)  at (10,0) {};
          \node (9)  at (-2,2) {};
          \node (10) at (-2,0) {};
          \node (11) at (12,0) {};
        \end{scope}

        \begin{scope}[every node/.style={fill=white, transform shape}]

          \begin{scope}[every edge/.style={draw}]
            \path (6)  edge[bend left=30] node {$0$} (7);
            \path (8)  edge node {$0$} (11);
            \path (1)  edge[bend right=30] node {$1$} (2);
            \path (3)  edge node {$1$} (4);
            \path (5)  edge node {$1$} (6);
            \path (7)  edge node {$1$} (8);
            \path (1)  edge node {$2$} (9);
            \path (2)  edge node {$2$} (10);
            \path (4)  edge node {$2$} (5);
            \path (6)  edge[bend right=30] node {$2$} (7);
            \path (1)  edge node {$3$} (3);
            \path (2)  edge node {$3$} (4);
            \path (1)  edge[bend left=30] node {$4$} (2);
            \path (9)  edge node {$4$} (10);
          \end{scope}
        \end{scope}

      \end{tikzpicture}
    \end{center}
  \end{figure}

  Nous allons maintenant essayer de prouver que les générateurs de ces graphes ne sont pas indépendants.

\end{frame}

\begin{frame}{Premier graphe}

  Reprenons le premier graphe.

  \begin{figure}[H]
    \begin{center}
      \begin{tikzpicture}[scale=.55]

        \begin{scope}[every node/.style={circle,draw, transform shape}]
          \node (1)  at (0,2)  {};
          \node (2)  at (0,0)  {};
          \node (3)  at (2,2)  {};
          \node (4)  at (2,0)  {};
          \node (5)  at (4,2)  {};
          \node (6)  at (4,0)  {};
          \node (7)  at (6,0)  {};
          \node (8)  at (8,2)  {};
          \node (9)  at (8,0)  {};
          \node (10) at (10,2) {};
          \node (11) at (10,0) {};
        \end{scope}

        \begin{scope}[every node/.style={fill=white, transform shape}]

          \begin{scope}[every edge/.style={draw}]
            \path (9)  edge node {$0$} (11);
            \path (8)  edge[bend left=30] node {$0$} (10);
            \path (1)  edge[bend right=30] node {$1$} (2);
            \path (3)  edge node {$1$} (4);
            \path (5)  edge node {$1$} (6);
            \path (7)  edge node {$1$} (9);
            \path (1)  edge[bend left=30] node {$2$} (3);
            \path (6)  edge node {$2$} (7);
            \path (8)  edge node {$2$} (9);
            \path (10) edge node {$2$} (11);
            \path (1)  edge[bend left=30] node {$3$} (2);
            \path (3)  edge node {$3$} (5);
            \path (4)  edge node {$3$} (6);
            \path (8)  edge[bend right=30] node {$3$} (10);
            \path (1)  edge[bend right=30] node {$4$} (3);
            \path (2)  edge node {$4$} (4);
          \end{scope}
        \end{scope}

      \end{tikzpicture}
    \end{center}
  \end{figure}

  Conservons uniquement les arêtes $\rho_1$, $\rho_2$ et $\rho_4$.

  \begin{figure}[H]
    \begin{center}
      \begin{tikzpicture}[scale=.55]

        \begin{scope}[every node/.style={circle,draw, transform shape}]
          \node (1)  at (0,2)  {};
          \node (2)  at (0,0)  {};
          \node (3)  at (2,2)  {};
          \node (4)  at (2,0)  {};
          \node (5)  at (6,0)  {};
          \node (6)  at (4,0)  {};
          \node (7)  at (8,0)  {};
          \node (8)  at (10,0)  {};
          \node (9)  at (12,0)  {};
          \node (10) at (14,0)  {};
          \node (11) at (16,0) {};
        \end{scope}

        \begin{scope}[every node/.style={fill=white, transform shape}]

          \begin{scope}[every edge/.style={draw}]
            \path (1)  edge node {$1$} (2);
            \path (3)  edge node {$1$} (4);
            \path (5)  edge node {$1$} (6);
            \path (7)  edge node {$1$} (8);
            \path (1)  edge[bend left=30] node {$2$} (3);
            \path (5)  edge node {$2$} (7);
            \path (8)  edge node {$2$} (9);
            \path (10) edge node {$2$} (11);
            \path (1)  edge[bend right=30] node {$4$} (3);
            \path (2)  edge node {$4$} (4);
          \end{scope}
        \end{scope}

      \end{tikzpicture}
    \end{center}
  \end{figure}

  \pause

  Il apparaît clairement que $\rho_4 = (\rho_1 \rho_2)^{10}$.

  Donc la propriété d'intersection ne peut être vérifiée.

\end{frame}

\begin{frame}{Second graphe (1)}

  Dans le cas du second graphe.

  \begin{figure}[H]
    \begin{center}
      \begin{tikzpicture}[scale=.55]

        \begin{scope}[every node/.style={circle,draw, transform shape}]
          \node (1)  at (0,2)  {};
          \node (2)  at (0,0)  {};
          \node (3)  at (2,2)  {};
          \node (4)  at (2,0)  {};
          \node (5)  at (4,0)  {};
          \node (6)  at (6,0)  {};
          \node (7)  at (8,0)  {};
          \node (8)  at (10,0) {};
          \node (9)  at (-2,2) {};
          \node (10) at (-2,0) {};
          \node (11) at (12,0) {};
        \end{scope}

        \begin{scope}[every node/.style={fill=white, transform shape}]

          \begin{scope}[every edge/.style={draw}]
            \path (6)  edge[bend left=30] node {$0$} (7);
            \path (8)  edge node {$0$} (11);
            \path (1)  edge[bend right=30] node {$1$} (2);
            \path (3)  edge node {$1$} (4);
            \path (5)  edge node {$1$} (6);
            \path (7)  edge node {$1$} (8);
            \path (1)  edge node {$2$} (9);
            \path (2)  edge node {$2$} (10);
            \path (4)  edge node {$2$} (5);
            \path (6)  edge[bend right=30] node {$2$} (7);
            \path (1)  edge node {$3$} (3);
            \path (2)  edge node {$3$} (4);
            \path (1)  edge[bend left=30] node {$4$} (2);
            \path (9)  edge node {$4$} (10);
          \end{scope}
        \end{scope}

      \end{tikzpicture}
    \end{center}
  \end{figure}

  Retirons les arêtes $\rho_4$.

  \begin{figure}[H]
    \begin{center}
      \begin{tikzpicture}[scale=.55]

        \begin{scope}[every node/.style={circle,draw, transform shape}]
          \node (1)  at (0,2)  {};
          \node (2)  at (0,0)  {};
          \node (3)  at (2,2)  {};
          \node (4)  at (2,0)  {};
          \node (5)  at (4,0)  {};
          \node (6)  at (6,0)  {};
          \node (7)  at (8,0)  {};
          \node (8)  at (10,0) {};
          \node (9)  at (-2,2) {};
          \node (10) at (-2,0) {};
          \node (11) at (12,0) {};
        \end{scope}

        \begin{scope}[every node/.style={fill=white, transform shape}]

          \begin{scope}[every edge/.style={draw}]
            \path (6)  edge[bend left=30] node {$0$} (7);
            \path (8)  edge node {$0$} (11);
            \path (1)  edge node {$1$} (2);
            \path (3)  edge node {$1$} (4);
            \path (5)  edge node {$1$} (6);
            \path (7)  edge node {$1$} (8);
            \path (1)  edge node {$2$} (9);
            \path (2)  edge node {$2$} (10);
            \path (4)  edge node {$2$} (5);
            \path (6)  edge[bend right=30] node {$2$} (7);
            \path (1)  edge node {$3$} (3);
            \path (2)  edge node {$3$} (4);
          \end{scope}
        \end{scope}

      \end{tikzpicture}
    \end{center}
  \end{figure}

  Nous allons prouver que ce graphe génère $A_{11}$ et donc que $\rho_4$ peut être exprimé comme un produit des autres générateurs.

\end{frame}

\begin{frame}{Second graphe (2)}
  \begin{figure}[H]
    \begin{center}
      \begin{tikzpicture}[scale=.55]

        \begin{scope}[every node/.style={circle,draw, transform shape}]
          \node (1)  at (0,2)  {};
          \node (2)  at (0,0)  {};
          \node (3)  at (2,2)  {};
          \node (4)  at (2,0)  {};
          \node (5)  at (4,0)  {};
          \node (6)  at (6,0)  {};
          \node (7)  at (8,0)  {};
          \node (8)  at (10,0) {};
          \node (9)  at (-2,2) {};
          \node (10) at (-2,0) {};
          \node (11) at (12,0) {};
        \end{scope}

        \begin{scope}[every node/.style={fill=white, transform shape}]

          \begin{scope}[every edge/.style={draw}]
            \path (6)  edge[bend left=30] node {$0$} (7);
            \path (8)  edge node {$0$} (11);
            \path (1)  edge node {$1$} (2);
            \path (3)  edge node {$1$} (4);
            \path (5)  edge node {$1$} (6);
            \path (7)  edge node {$1$} (8);
            \path (1)  edge node {$2$} (9);
            \path (2)  edge node {$2$} (10);
            \path (4)  edge node {$2$} (5);
            \path (6)  edge[bend right=30] node {$2$} (7);
            \path (1)  edge node {$3$} (3);
            \path (2)  edge node {$3$} (4);
          \end{scope}
        \end{scope}

      \end{tikzpicture}
    \end{center}
  \end{figure}

  Ce graphe est 2-transitif, il est donc primitif. Dès lors il est possible d'utiliser la classification des groupes primitifs afin de prouver que ce graphe génère $A_{11}$. Le sous-groupe généré par $\rho_2$ et $\rho_3$ est $D_{24}$, l'ordre du groupe doit donc être un multiple de 1320.

  Il y a trois possibilités pour un tel ordre: $M_{11}$, $A_{11}$ ou $S_{11}$. La première est impossible car $M_{11}$ n'a pas $D_{24}$ comme sous-groupe et $S_{11}$ est impossible car tous les générateurs sont des permutations paires. Ce graphe doit donc générer $A_{11}$.

\end{frame}

\section{$\rho_2$ ne peut être une 4-transposition}



\end{document}

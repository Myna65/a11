\documentclass[10pt]{beamer}

\usepackage[utf8]{inputenc}
\usepackage[french]{babel}

\usetheme[progressbar=frametitle]{metropolis}
\usepackage{appendixnumberbeamer}

\usepackage{booktabs}
\usepackage[scale=2]{ccicons}

\usepackage{pgfplots}
\usepgfplotslibrary{dateplot}

\usepackage{xspace}
\newcommand{\themename}{\textbf{\textsc{metropolis}}\xspace}

\usepackage{tikz}
\usepackage{svg}

\usepackage{amsthm}
\usepackage{amsfonts}
\usepackage{mathrsfs}

\usetikzlibrary{graphs}
\usetikzlibrary{graphs.standard}

\title{Abstract polytopes on $A_{11}$}
%\subtitle{Graphes de Moore}
% \date{\today}
\date{}
\author{Nathan Meynaert}
% \titlegraphic{\hfill\includegraphics[height=1.5cm]{logo.pdf}}

\begin{document}

\maketitle

\begin{frame}{Table des matières}
  \setbeamertemplate{section in toc}[sections numbered]
  \tableofcontents[hideallsubsections]
\end{frame}

\section{Polytopes abstraits}

\begin{frame}{Définition}

  \begin{definition}
    Un \textit{polytope abstrait} est un ensemble partiellement ordonné, gradué,  dont les éléments sont appelés \textit{faces} et sont ordonnés par l'inclusion $\subseteq$ tels que
    \begin{itemize}
      \item \textbf{(P1)} $\mathcal P$ contient deux faces impropres: $F_{-1}$ et $F_d$.
      \item \textbf{(P2)} Les drapeaux de $\mathcal P$ contiennent $d+2$ faces (en incluant les deux faces impropres)    \item \textbf{(P3)} $\mathcal P$ est strictement connecté.
      \item \textbf{(P4)} $\mathcal P$ satisfait la propriété du diamant.
    \end{itemize}
  \end{definition}

\end{frame}

\begin{frame}{Strictement connecté}
\end{frame}

\begin{frame}{4-transposition}
\end{frame}

\section{Permutation representation graphs}

\begin{frame}
  Un permutation representation graph est un graphe représentation un sggi. Le sggi est composé de générateurs $\rho_0, \rho_{r-1}$. Ce graph est un multigraphe avec labels dont les sommets sont les points du string C-group. Deux points $\alpha$ et $\beta$ sont reliés par une arête avec un label $i$ ssi $\alpha \rho_i = \beta$. Par convention, le graphe ne contient pas de boucle.


  \pause

  \begin{figure}[H]
    \begin{center}
      \begin{tikzpicture}[scale=.8]

        \begin{scope}[every node/.style={circle,draw, transform shape}]
          \node (1)  at (6,0)  {};
          \node (2)  at (6,2)  {};
          \node (3)  at (4,2)  {};
          \node (4)  at (4,0)  {};
          \node (5)  at (4,-2) {};
          \node (6)  at (2,-2) {};
          \node (7)  at (2,0)  {};
          \node (8)  at (2,2)  {};
          \node (9)  at (12,0) {};
          \node (10) at (10,0) {};
          \node (11) at (8,0) {};
        \end{scope}

        \begin{scope}[every node/.style={fill=white, transform shape}]

          \begin{scope}[every edge/.style={draw}]
            \path (1)  edge node {$0$} (2);
            \path (3)  edge node {$0$} (4);
            \path (5)  edge[bend right=50] node {$0$} (6);
            \path (7)  edge node {$0$} (8);
            \path (1)  edge node {$1$} (11);
            \path (3)  edge[bend right=30] node {$1$} (8);
            \path (4)  edge node {$1$} (5);
            \path (6)  edge node {$1$} (7);
            \path (1)  edge node {$2$} (4);
            \path (2)  edge node {$2$} (3);
            \path (5)  edge node {$2$} (6);
            \path (10) edge node {$2$} (11);
            \path (3)  edge[bend left=30] node {$3$} (8);
            \path (4)  edge node {$3$} (7);
            \path (5)  edge[bend left=50] node {$3$} (6);
            \path (9)  edge node {$3$} (10);
          \end{scope}
        \end{scope}

      \end{tikzpicture}
    \end{center}
  \end{figure}
\end{frame}

\begin{frame}{Motifs possibles}
  %Les générateurs non consecutifs d'un sggi doivent commuter, ceci a des conséquences sur le graphe.

  \begin{theorem}
    Dans un permutation representation graph, le graphe obtenu en ne conservant que les arêtes de deux générateurs non-consécutifs peuvent uniquement former les motifs suivants:
    \begin{itemize}
      \item Un point fixe
      \item Une arête simple
      \item Une arête double
      \item Un carré alterné
    \end{itemize}
  \end{theorem}

  \pause

  \begin{figure}[H]
    \begin{center}
      \begin{tikzpicture}[scale=.55]

        \begin{scope}[every node/.style={circle,draw, transform shape}]
          \node (1)  at (0,0)  {};
          \node (2)  at (2,0)  {};
          \node (3)  at (4,0)  {};
          \node (4)  at (6,0)  {};
          \node (5)  at (8,0) {};
          \node (6)  at (10,0) {};
          \node (7)  at (12,0)  {};
          \node (8)  at (14,0)  {};
          \node (9)  at (14,2) {};
          \node (10) at (16,0) {};
          \node (11) at (16,2) {};
        \end{scope}

        \begin{scope}[every node/.style={fill=white, transform shape}]

          \begin{scope}[every edge/.style={draw}]
            \path (2)  edge node {$i$} (3);
            \path (6)  edge[bend left=30] node {$i$} (7);
            \path (8)  edge node {$i$} (10);
            \path (9)  edge node {$i$} (11);
            \path (4)  edge node {$j$} (5);
            \path (6)  edge[bend right=30] node {$j$} (7);
            \path (8)  edge node {$j$} (9);
            \path (10) edge node {$j$} (11);
          \end{scope}
        \end{scope}

      \end{tikzpicture}
    \end{center}
  \end{figure}

\end{frame}

\begin{frame}{Exemple (1)}
  Un exemple:

  \begin{figure}[H]
    \begin{center}
      \begin{tikzpicture}[scale=.8]

        \begin{scope}[every node/.style={circle,draw, transform shape}]
          \node (1)  at (6,0)  {};
          \node (2)  at (6,2)  {};
          \node (3)  at (4,2)  {};
          \node (4)  at (4,0)  {};
          \node (5)  at (4,-2) {};
          \node (6)  at (2,-2) {};
          \node (7)  at (2,0)  {};
          \node (8)  at (2,2)  {};
          \node (9)  at (12,0) {};
          \node (10) at (10,0) {};
          \node (11) at (8,0) {};
        \end{scope}

        \begin{scope}[every node/.style={fill=white, transform shape}]

          \begin{scope}[every edge/.style={draw}]
            \path (1)  edge node {$0$} (2);
            \path (3)  edge node {$0$} (4);
            \path (5)  edge[bend right=50] node {$0$} (6);
            \path (7)  edge node {$0$} (8);
            \path (1)  edge node {$1$} (11);
            \path (3)  edge[bend right=30] node {$1$} (8);
            \path (4)  edge node {$1$} (5);
            \path (6)  edge node {$1$} (7);
            \path (1)  edge node {$2$} (4);
            \path (2)  edge node {$2$} (3);
            \path (5)  edge node {$2$} (6);
            \path (10) edge node {$2$} (11);
            \path (3)  edge[bend left=30] node {$3$} (8);
            \path (4)  edge node {$3$} (7);
            \path (5)  edge[bend left=50] node {$3$} (6);
            \path (9)  edge node {$3$} (10);
          \end{scope}
        \end{scope}

      \end{tikzpicture}
    \end{center}
  \end{figure}
\end{frame}

\begin{frame}{Exemple (2)}
  Un second exemple plus compliqué:

  \begin{figure}[H]
    \begin{center}
      \begin{tikzpicture}[scale=.8]

        \begin{scope}[every node/.style={circle,draw, transform shape}]
          \node (1)  at (0,2)  {h};
          \node (2)  at (0,0)  {g};
          \node (3)  at (0,-2) {f};
          \node (4)  at (-2,2)  {e};
          \node (5)  at (-2,0)  {d};
          \node (6)  at (-2,-2) {c};
          \node (7)  at (-4,2)  {b};
          \node (8)  at (-4,0)  {a};
          \node (9)  at (-2,4)  {i};
          \node (10) at (0,4)   {j};
          \node (11) at (2,0)   {k};
          \node (12) at (2,2)   {l};
        \end{scope}

        \begin{scope}[every node/.style={fill=white, transform shape}]

          \node (t1) at (-5,1)  {$i$};
          \node (t2) at (3,1)   {$i$};
          \node (t3) at (-1,-3) {$i+1$};
          \node (t4) at (-1,5)  {$i+1$};

          \begin{scope}[every edge/.style={draw}]
            \path (2)  edge node {$i+2$} (3);
            \path (5)  edge node {$i+2$} (6);
            \path (9)  edge node {$i+2$} (4);
            \path (10) edge node {$i+2$} (1);
            \path (1)  edge node {$i$} (2);
            \path (4)  edge node {$i$} (5);
            \path (7)  edge node {$i$} (8);
            \path (11) edge node {$i$} (12);
            \path (9)  edge[bend right=45] (t1);
            \path (t1) edge[bend right=45] (6);
            \path (10) edge[bend left=45] (t2);
            \path (t2) edge[bend left=45] (3);
            \path (12) edge[bend right=45] (t4);
            \path (t4) edge[bend right=45] (7);
            \path (11) edge[bend left=45] (t3);
            \path (t3) edge[bend left=45] (8);
            \path (1)  edge node {$i+1$} (4);
            \path (2)  edge node {$i+1$} (5);
            \path (3)  edge node {$i+1$} (6);
            \path (4)  edge node {$i+3$} (7);
            \path (5)  edge node {$i+3$} (8);
            \path (1)  edge node {$i+3$} (12);
            \path (2)  edge node {$i+3$} (11);
          \end{scope}
        \end{scope}

      \end{tikzpicture}
    \end{center}
  \end{figure}
\end{frame}

\section{Nombre minimal de 4-transpositions}

\begin{frame}{Rang 5}
  \begin{theorem}
    Tout permutation representation graph de rang 5 de $A_{11}$ admet au moins une 4-transposition.
  \end{theorem}
\end{frame}

\begin{frame}{Preuve (1)}
    Supposons qu'il n'y ait aucune 4-transposition alors tous les générateurs doivent être des 2-transpositions et donc il y a 10 arêtes pour lier 11 points. Le graphe doit donc être un arbre.

    Cependant aucun sommet ne peut être relié par 3 arêtes.

    \begin{figure}[H]
      \begin{center}
        \begin{tikzpicture}[scale=.8]

          \begin{scope}[every node/.style={circle,draw, transform shape}]
            \node (1)  at (0,0)  {};
            \node (2)  at (2,0)  {};
            \node (3)  at (4,0)  {};
            \node (4)  at (2,2)  {};
          \end{scope}

          \begin{scope}[every node/.style={fill=white, transform shape}]
            \begin{scope}[every edge/.style={draw}]
              \path (1)  edge node {$i$} (2);
              \path (2)  edge node {$j$} (3);
              \path (2)  edge node {$k$} (4);
            \end{scope}
          \end{scope}

        \end{tikzpicture}
      \end{center}
    \end{figure}

    Donc le graphe doit être une chaîne.
\end{frame}

\begin{frame}{Preuve (2)}

\end{frame}

\section{Ni $\rho_0$ ni $\rho_4$ ne peut être une 4-transposition}

\section{Ni $\rho_1$ ni $\rho_3$ ne peut être une 4-transposition}

\section{$\rho_2$ ne peut être une 4-transposition}



\end{document}

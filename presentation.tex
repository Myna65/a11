\documentclass[10pt]{beamer}

\usepackage[utf8]{inputenc}
\usepackage[french]{babel}

\usetheme[progressbar=frametitle]{metropolis}
\usepackage{appendixnumberbeamer}

\usepackage{booktabs}
\usepackage[scale=2]{ccicons}

\usepackage{pgfplots}
\usepgfplotslibrary{dateplot}

\usepackage{xspace}
\newcommand{\themename}{\textbf{\textsc{metropolis}}\xspace}

\usepackage{tikz}
\usepackage{svg}

\usepackage{amsthm}
\usepackage{amsfonts}
\usepackage{mathrsfs}

\usetikzlibrary{graphs}
\usetikzlibrary{graphs.standard}

\title{Abstract polytopes on $A_{11}$}
%\subtitle{Graphes de Moore}
% \date{\today}
\date{}
\author{Nathan Meynaert}
% \titlegraphic{\hfill\includegraphics[height=1.5cm]{logo.pdf}}

\begin{document}

\maketitle

\begin{frame}{Table des matières}
  \setbeamertemplate{section in toc}[sections numbered]
  \tableofcontents[hideallsubsections]
\end{frame}

\section{Polytopes abstraits}

\begin{frame}{Définition}

  \begin{definition}
    Un \textit{polytope abstrait} est un ensemble partiellement ordonné gradué  dont les éléments sont appelés \textit{faces} et sont ordonnés par l'inclusion $\subseteq$ tels que
    \begin{itemize}
      \item \textbf{(P1)} Il existe un face maximale $F_d$ et une face minimale $F_{-1}$.
    \end{itemize}
  \end{definition}

\end{frame}

\end{document}
